\documentclass{beamer}

\mode<presentation> {
\usetheme{Madrid}
\usecolortheme{beaver}
%\usecolortheme{orchid}
%\usecolortheme{wolverine}

%\setbeamertemplate{footline} % To remove the footer line in all slides uncomment this line
\setbeamertemplate{footline}[page number] % To replace the footer line in all slides with a simple slide count uncomment this line

\setbeamertemplate{navigation symbols}{} % To remove the navigation symbols from the bottom of all slides uncomment this line
}

\usepackage{graphicx}
\usepackage{booktabs}
\usepackage{moresize}

%\usepackage {xcolor}
\definecolor {processblue}{cmyk}{0.96,0,0,0}
\definecolor {upmBlue}{rgb}{0.2, 0.6745, 1}

% -------------
% BEGIN - TITLE
% -------------

\title[Short title]{Proyecto de Inteligencia Artificial}

\author{Rubén Aguado Cosano - z170284 \\ Younes Idrissi Boulid - z170155 \\ Paula Pousa Martinez - z170068 \\ Jorge Sol Gonzalez - z170212}

\institute[ETSIINF UPM] 
{
  \textcolor{upmBlue}{Universidad Politecnica de Madrid}\\
\medskip
}
\date{\today}

\begin{document}

\begin{frame}
\titlepage
\end{frame}

\begin{frame}
\frametitle{Indice}
\tableofcontents
\end{frame}

% -----------
% END - TITLE
% -----------

%	BEGIN - PRESENTATION
%---------------------
\section{Arquitectura del Proyecto}
\begin{frame}{Arquitectura del Proyecto}
  \begin{figure}[H]
    \centering
    \includegraphics[scale=0.40]{"../pics/diagrama"}
  \end{figure}
\end{frame}

\section{Recogida de Datos}
\begin{frame}{Recogida de Datos}
    \begin{itemize}
      \item La magnitud de medida son los \textbf{metros}.
      \item Para la obtención de las \textbf{coordenadas} de las paradas se utilizó Google Maps.
        \begin{figure}[H]
         \centering
         \includegraphics[scale=0.25]{"../pics/coordenadas"}
        \end{figure}
    \end{itemize}
\end{frame}

\begin{frame}{Recogida de Datos (Formula de Haversine)}
  \begin{figure}[H]
    \centering
    \includegraphics[scale=0.20]{"../pics/haversine"}
  \end{figure}
\end{frame}

\begin{frame}{Recogidas de Datos}
    La recogida de las distancias reales se realizó a mano con ayuda de la página web \textbf{HyperDia}
    \vspace{0.5cm}
    \begin{figure}[H]
      \centering
      \includegraphics[scale=0.30]{"../pics/hyperDia"}
    \end{figure}
\end{frame}

\section{Base de Datos}
\begin{frame}{Base de Datos}
  \begin{figure}[H]
    \centering
    \includegraphics[scale=0.35]{"../pics/umlBDD"}
  \end{figure}
  \vspace{0.3cm}
  \begin{figure}[H]
    \centering
    \includegraphics[scale=0.37]{"../pics/printBDD"}
  \end{figure}
\end{frame}

\begin{frame}{Base de Datos}
  \begin{figure}[H]
    \centering
    \includegraphics[scale=0.30]{"../pics/querys"}
  \end{figure}
\end{frame}

\section{Algoritmo A Estrella}
\begin{frame}{Algoritmo A Estrella}
  \begin{figure}[H]
    \centering
    \includegraphics[scale=0.23]{"../pics/aEstrella"}
  \end{figure}
\end{frame}

\section{Demo}
\begin{frame}{Demo}
  \begin{figure}[H]
    \centering
    \includegraphics[scale=0.25]{"../pics/web"}
  \end{figure}

  \begin{center}
    {\ssmall Rubén Aguado Cosano - z170284 \\ 
      Younes Idrissi Boulid - z170155 \\ 
      Paula Pousa Martinez - z170068 \\ 
      Jorge Sol Gonzalez - z170212 \\
    }
  \end{center}
\end{frame}

%---------------------
%	END - PRESENTATION
%---------------------
\end{document}
